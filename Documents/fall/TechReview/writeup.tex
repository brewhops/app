% Technology Review
% CS 461 - CS Senior Capstone
% Fall 2017
% Authors: Connor Christensen, Lily Shellhammer, William Buffum


\documentclass[draftclsnofoot,onecolumn,letterpaper,10pt,compsoc]{IEEEtran}

% Packaging
\usepackage{geometry}
\usepackage{hyperref}
\usepackage{titling}
\usepackage{color}
\usepackage{listings}
\usepackage{cite}
\usepackage{pdfpages}
\usepackage{pdflscape}
\usepackage{url}
\usepackage{array}
\usepackage{xargs}                      % Use more than one optional parameter in a new commands


\usepackage[pdftex,dvipsnames]{xcolor}  % Coloured text etc.
\usepackage[colorinlistoftodos,prependcaption,textsize=tiny]{todonotes}
\newcommandx{\note}[2][1=]{\todo[linecolor=red,backgroundcolor=red!25,bordercolor=red,#1]{#2}}

\newcommand\question[1]{\footnote{\textcolor{red}{#1}}}



\definecolor{lightgray}{rgb}{.9,.9,.9}
\definecolor{darkgray}{rgb}{.4,.4,.4}
\definecolor{purple}{rgb}{0.65, 0.12, 0.82}

\lstdefinelanguage{JavaScript}{
  keywords={typeof, new, true, false, catch, function, return, null, catch, switch, var, if, in, while, do, else, case, break},
  keywordstyle=\color{blue}\bfseries,
  ndkeywords={class, export, boolean, throw, implements, import, this},
  ndkeywordstyle=\color{darkgray}\bfseries,
  identifierstyle=\color{black},
  sensitive=false,
  comment=[l]{//},
  morecomment=[s]{/*}{*/},
  commentstyle=\color{purple}\ttfamily,
  stringstyle=\color{red}\ttfamily,
  morestring=[b]',
  morestring=[b]"
}

\lstset{
   language=JavaScript,
   extendedchars=true,
   basicstyle=\footnotesize\ttfamily,
   showstringspaces=false,
   showspaces=false,
   tabsize=2,
   breaklines=true,
   showtabs=false,
   captionpos=b,
   basicstyle=\tiny
}



% Paper type
\geometry{letterpaper, margin=.75in}

% Title page
\title{CS 461 - CS Senior Capstone
	\\Fall 2017
	\\Technology Review
}


\author{
	Connor I. Christensen \\
	\texttt{chriconn@oregonstate.edu}
	\\
	Lily M. Shellhammer \\
	\texttt{shellhal@oregonstate.edu}
	\\
	William B. Buffum \\
	\small{}
	\texttt{buffumw@oregonstate.edu}
}

\begin{document}

\begin{titlingpage}
    \maketitle
    \begin{abstract}
			Ninkasi Brewing Company is based in Eugene, Oregon, producing and distributing nearly 100,000 barrels of beer each year across the United States and Canada.
			Ninkasi currently tracks brewery data using digital spreadsheets, a laborious, time consuming, and error prone process.
			Quality brewing requires the company to be detail-oriented, organize its data and provide timely actions in the brewing process.
			In order to maintain good quality control in their product and give the company room to scale in its production, our team has been tasked with creating software that will improve the process of entering, storing and accessing data related to the brewing process.
			This document examines several technologies that could be useful to the construction, maintenance, speed and reliability of software requested by Ninkasi.
			After weighing the benefits and drawbacks of each technology, a decision is made on the best choice of technology to fit the requirements of the client.
			\\
			\textbf{Keywords:} Brewing, Operations, Management
    \end{abstract}
		\pagebreak
		\tableofcontents
\end{titlingpage}

\section{Introduction}

  \subsection{Front-End}
  This is a comparison of three front end technologies for web design.
  \begin{itemize}
    \item Data visualization seeks to find a framework that will aid in the display of information from the database in a user-friendly way.
    \item Styling attempts to cover available options for creating a nice user interface for the project.
    \item Interactive web frameworks outlines the difference between the major pro's and con's of JavaScript frameworks and whether or not they should be used at all.
  \end{itemize}

  \textbf{Warning:}
  The majority of technologies listed in these categories is open source code that builds off other open source modules.
  Including code from other open source projects could potentially lead to legal issues.
  Each module added into the project comes with a legal agreement that could potentially cause problems for the company later.
  These legal agreements may or may not be enforced and they may or may not conflict with other modules in the project.
  It is common web development practice to build frameworks from many sub-utilities and packages that are available in the open source community.
  As of yet, there is no dedicated team ensuring that all legal agreements agree with each-other and the requirements of the client, and it is up to the client to decide if they want to take steps to guard against the unlikely event that someone stakes a claim against the company based on those licenses.

  \subsection{Connections}
  My job in the Ninkasi BrewHops project is to develop and organize the middle of the stack: the languages that connect the database to the front end.
  I will be coding the database interaction languages and the functionality to pull information from/put information into the database.
  In this document, I outline three different possibilities we could use in our project for hosting, platform, and database interaction language possibilities.

  \subsection{Back-End}
  In this technology review, we will analyze the costs and benefits associated with various tools in three components of our project.
  First, we will look at peripheral interfaces.
  It is important to understand how brewers will access information managed by our system and recognize the positive and negative attributes of each option that exists.
  Next, we will look at database technologies.
  It is important to look at the structure of our data and understand the costs and benefits of choosing a certain technology.
  Finally, task runners and module bundlers are an important tool in any efficient development process.
  Looking at the different options and determining which is a good fit is important before the BrewHops team can effectively develop a software solution for Ninkasi Brewing Company.
  Overall, each technology conveys pros and cons for its usage; in some scenarios there is no perfect answer, but understanding why certain technologies exist will help improve the development of our data management system.

\section{Data Visualization}
  \subsection{Introduction}
  This is an overview of libraries written for data display in web browsers.
  Each of these libraries are built using the web language JavaScript to allow for live interaction with the user.
  Table 1 shows the code needed to set up a basic bar graph using each library.
  These yield approximately the same visual results.
  All have the con that they require JavaScript to be enabled and all of them are open source and free.

  \subsection{Critera}
  \begin{itemize}
    \item relatively small amounts of code
    \item stable and consistent in its presentation
    \item provide simple charts and graphs like bar charts and plot point graphs
    \item easy for developers to work with
  \end{itemize}

  \subsection{Technology}

    \subsubsection{D3.js}

    \textbf{About:}

    D3 stands for Data-Driven Documents and it is a powerhouse of a library for data visualization.\cite{d3.org}
    It was created by a team of PhD graduates working out of the Stanford Visualization Group and is essentially the library for creating amazing data visualization.\cite{d3Journal}
    It has a relatively bulky setup, but allows fine grained detail and control over how data is displayed.
    This library is so powerful that it functions more as a library for painting than for graphing.

    \noindent \textbf{Pros:}

    D3 uses SVG's \footnote{\textbf{S}calable \textbf{V}ector \textbf{G}raphics: A visual component defined mathematically. It can scale up or down to any size without loss of image quality.} which allows it to be able to create any 2d shape that can be mathematically defined.
    SVG's do not require JavaScript to work, though D3 can't run without JavaScript enabled.
    The D3 library uses pre-build JavaScript functions to select elements, create SVG objects, and be able to preform a wide number of transformations on them.\cite{d3.org}
    Transformations are just as precise as the drawing abilities and are limited only by the processing power of the computer and the skill of the programmer.
    It uses jQuery and CSS styled selection and modification of content for flexibility and ease of use.
    Objects that are made with D3 are easily syllable with CSS, meaning that data display can inherit styling rules to maintain a consistent look across the platform.
    D3 has a large community and it was built by very intelligent people early on, so it has only gained in popularity since then.\cite{DataVisProCon}
    The majority of data frameworks are built off of D3.
    As such, there are a huge number of online examples.

    \noindent \textbf{Cons:}

    Despite all its good features, the D3 library adds a lot of code to a project, the learning curve is steep, and the code needed to accomplish a task is verbose.\cite{DataVisProCon}
    The code required to set up a bar graph is massive compared to most other frameworks and the time that it takes to learn D3 is an even bigger obstacle.

    \subsubsection{C3.js}
    \textbf{About:}

    C3 is a package built off D3 and is formatted specifically for creating graphs.

    \noindent \textbf{Pros:}

    It is beautifully simple to create a graph and plug and play really works in this context.
    It has a great set of examples and documentation is readily available.
    It can easily switch between chart types and display multiple chart types mixed in with a single variable changed in the code.\cite{c3.org}
    The amount of things that just work coming out of C3 is impressive, inclusive of the animations that simply show up whenever you load up a graph.

    \noindent \textbf{Cons:}

    Since it is built off D3, C3 requires D3 to be installed, which causes many lines of code to be added to the project.
    There are nearly 10,000 lines of JavaScript in both D3 and C3, bringing in almost 20,000 lines of code for data display.
    The interface is much easier to use, the tradeoff being more limiting in its expressions.


    \subsubsection{Chart.js}
    \textbf{About:}
    Chart.js is a JavaScript library that allows you to draw different types of charts using the HTML5 canvas element
    \footnote{HTML Canvas: A new web standard that allows web programmers to create computer graphics created and rendered in the browser.}.

    \noindent \textbf{Pros:}

    Chart.js does not use D3 as its underlying code, which means that it is significantly more lightweight as a package.\cite{ChartJS} Like C3, it is very responsive and the documentation is very good.

    \noindent \textbf{Cons:}

    Use of the canvas comes with a few drawbacks, the most common issue being that it cannot scale without loss of quality.
    The HTML5 Canvas specification recommends that authors should not use the canvas element when they have other more suitable means available.\cite{CanvasVsSVG}
    Canvas is good for 3d graphics, but this ability is not beneficial if you simply want a bar graph.
    If you are drawing little details all very close together, canvas is great for that.
    Canvas is not very accessible as it is just drawing pixels and no data can be extracted by assistive technology or bots.

  \subsection{Conclusion}
  It is on our recommendation that C3 be the framework of choice.
  D3 can produce some seriously impressive data visualizations, and chart.js is small and simple, but for the purposes of this project, C3 fits the criteria best.
  As seen in Table 1 in the appendix, the amount of code it takes to create a bar graph is very concise and easy to use, which makes development and maintenance straightforward and produce a small amount of bugs.
  The results are beautiful, informative and user friendly.
  Given the scope of the project, being able to utilize SVG graphing technology in a straightforward and simple way will enhance the product.

\section{Styling}

  \subsection{Introduction}
  Styling determines how the content is laid out on the page.
  This is what makes layout usable on a variety of screen sizes and the content user friendly.
  There are several methods to apply styling for a webpage, either by building your own, or using a framework that other people have written and dropping your content into their code.

  \subsection{Criteria}
  \begin{itemize}
    \item easily modifiable and maintainable
    \item present a small amount of UI bugs
    \item run quickly
  \end{itemize}

  \subsection{Technology}
    \subsubsection{CSS}

    \textbf{About:}

    Apart from some technologies like SVG's and the HTML5 canvas, CSS is responsible for all website styling.
    It was invented in 1996 and has been one of the three major web languages since then.
    Any method in this styling section is using CSS at some level to deliver it's product.\cite{CSSHistory}

    \noindent \textbf{Pros:}

    Raw CSS is the de facto standard for styling websites.
    There is no other method for changing text color, aligning content on a page, adding drop shadows, etc. other than CSS.
    It is known by all web developers and the community is massive.
    Every bit of code written for styling the web, uses CSS at some point, and because the web always makes client side code like CSS visible to the user, anything you can see is an example you can follow.
    It is easy to debug when you put raw CSS straight into the browser, and if you are writing CSS, you can simply drop that into a browser and it will run without any extra effort.\cite{CSSProCon}

    \noindent \textbf{Cons:}

    Many developers have moved away from writing raw CSS and use a preprocessor or framework, as it is easy to produce difficult to maintain code.
    CSS is much better than the alternative of writing all the styles into the HTML, but CSS still lacks some features that would make organization easier on developers.
    As such, developers working in CSS need to be highly organized if the project gets big enough, and the documentation they write must be clear so other developers can work from what they have built.
    CSS is a syntactically easy language, and in many cases, understanding it is intuitive.
    But there are components to CSS that are complicated and much less intuitive, generally having to do with layout.\cite{CSSProCon}
    CSS requires documentation for future engineers to quickly make sense of the complicated parts of CSS, as writing raw CSS means there is no framework to help standardize how CSS is written.
    For larger projects, CSS developers end up copying and pasting code frequently, which is a bad sign.
    This makes it much easier for inconsistent code to appear and makes it harder to change something on a wide scale.
    Say for example you want to change a single color across your site.
    In CSS, this requires finding every instance of that color and changing it to the new value.

    \subsubsection{Preprocessors}
    \textbf{About:}

    Preprocessors are programming languages that utilize some kind of compiler to translate the code into CSS to style web pages.
    They come with the benefit of being able to build new features to make it easier on developers without needing to consult with the W3C\footnote{Listed in the appendix}.
    This allows for flexible work environments which can lead to safer, more efficient and easier coding environments.
    The downside to this freedom is the cost of including another service standing in-between your coding and the finished product.
    It's possible that the preprocessor can develop or contain bugs, and its possible they stop supporting or developing the software.
    That being said, preprocessors have become powerful, widely understood and supported, and have become a tool useful for most projects.

    \noindent \textbf{Pros:}

    Developing with a preprocessor produces the same lightweight code for the user as if the developer had written it straight in CSS. This makes it easier on developers with no sacrifices for the users. Preprocessors provide some really great features like:
    \begin{itemize}
      \item Modular code abilities - developers can separate code into multiple files, which helps with organization. This also means they can include someone else's code into a project without directly pasting it into the custom code for the project.\cite{sass}
      \item Less redundancy in code - preprocessors support the ability to define functions and macros that can reduce the amount of code the developers have to write.
      \item Make it easy to make cascading changes - the ability to define variables makes for simple changes\footnote{CSS is implementing variables in its new standards now, however, this is still slower and not as well supported.}.
      \item Faster development compared to regular CSS - less typing and more organization makes it faster for developers to work.\cite{sass}
      \item Safer code - preprocessors usually automate the long and difficult process of making sure code is compatible with all browsers.
    \end{itemize}

    All the extra steps required by a preprocessor to compile before use are done before deploying, meaning that the client sees no slower performance as a result of the developers using a preprocessor.
    This gives the developers more freedom without a sacrifice for the user.

    \noindent \textbf{Cons:}

    Given all those nice pro's, preprocessors are still less well known than CSS.
    The different versions of preprocessors make it more difficult to find a whole development team that is already familiar with the language.
    In order for the developers to use whichever preprocessor they pick, this involves another thing they have to install to be able to get a website up and running.
    Installing a compiler also means that whatever code you write now relies on the compiler to be able to produce code that can be interpreted by browsers.


  	\subsubsection{Bootstrap}

    \noindent \textbf{Pros:}

    Bootstrap is an open source and free framework developed by twitter with a huge amount of users and documentation.\cite{bootstrap}
    It has been "battle tested" on thousands of website implementations and has a small amount of issues.
    It includes HTML, JS and CSS components built in, which allows for development magnitudes faster than writing the code by hand.
    Bootstrap is designed with mobile layouts in mind, and has the feature that your website will be compatible with all sorts of screens right off the bat.
    Bootstrap has specified layouts, buttons, menus and icons that they want you to use which is limiting, but means a development team can produce something that looks good without needing anyone who understands graphic design.
    It is standardized and many developers know the framework, so maintenance and development by other teams is easy.

    \noindent \textbf{Cons:}

    Bootstrap is a really helpful but very big framework.
    There are tens of thousands of lines of code included in the project, and it also requires jQuery, which further inflates the size of the framework.
    This makes sites slower, heavier, and if you want to customize any elements in the page, you have to overwrite the CSS rules in Bootstrap.\cite{BootstrapProCon}
    This can be a painful process as it is a bad coding practice to put more code into a project to cancel out previous code.
    It leads to bugs and weird visual issues that would not show up with good development from scratch.
    With allowing the users to write less CSS, it requires that they offload page content, styling and functionality into the HTML, which can create bulky and illegible code for the content of your page.
    The drawback of having a framework make the stylistic decisions for you is that all Bootstrap websites end up looking the same.

  \subsection{Conclusion}
  Preprocessors are generally considered the best option for projects that want the benefit of being able to create custom interfaces without having to deal with the drawbacks of writing raw CSS.
  Sass is one of the big three preprocessors, which offers a lot of flexibility in how you write the code, and really speeds up the development process.
  If for some reason the next group decides that they want to work just with CSS, they can always compile the CSS and then work with that from then on.

\section{Interactive Web Frameworks}
  \subsection{Introduction}
  There are many JavaScript frameworks out there for creating interactive websites.
  In fact, there are so many that the biggest issue becomes which one to choose, rather than whether to use it or not.
  They have built in functionality for a wide variety of things that are rather complicated to with native JavaScript.
  These JavaScript frameworks are so popular that nearly every major tech company has created their own and were kind enough to make them open source.

  \subsection{Critera}
  \begin{itemize}
    \item small
    \item fast
    \item well maintained as a framework
    \item make data manipulation easy for the developers
  \end{itemize}


  \subsection{Pros and cons for all JavaScript frameworks}
  \textbf{Pros:}
  \begin{itemize}
    \item Responsive websites - It is easy for the user to interact with components and get feedback.
    \item Much faster than developing it all from scratch.
    \item Frequently have the ability to extend the framework with plugins.
    \item Uses Model-View-Controller philosophy - This is a practice that helps developers keep a separation of states for data in the website. It leads to more controllable and better coding.
  \end{itemize}

  \noindent \textbf{Cons:}
  \begin{itemize}
    \item They all have a somewhat steep learning curve.
    \item There are so many of them it is harder to find a developer that is familiar with even just the popular frameworks
    \item If you are building a really tiny web app, the use of a JavaScript framework  can slow down the site
  \end{itemize}

  \subsection{Technology}
  	\subsubsection{Angular}

    \textbf{About:}

    AngularJS was originally developed in 2009 by Misko Hevery\cite{AngularIntroduction}.
    The original intent of the project to be an end-to-end tool that allowed web designers to interact with both the frontend and the backend.\cite{HistoryOfAngular}
    Hevery began working at Google and his project was noticed by the company and has been developed by Google employees since then.

    \noindent \textbf{Pros:}

    Angular was created early in the age of JavaScript frameworks and was one of the first systems of its kind.
    It is currently being maintained by Google, and having the backing of such a big company means that it will be sticking around.
    It has a huge user base and it is easy to find someone that has experience in developing with it.

    \noindent \textbf{Cons:}

    It is one of the oldest of its kind.
    It didn't have other frameworks come first to learn what mistakes not to make.
    As such, Angular has tried to reinvent itself several times and versioning of Angular is really confusing. Each version is so different it could be considered a different framework.
    Angular 1 is really slow\cite{SpeedReport} and can get messy really easily.


  	\subsubsection{React}
    \textbf{About:}

    React allows developers to create large web-applications that use data and can change over time without reloading the page.
    It uses a similar ideology to Angular in use of the Model-View-Controller, but it is different in its organization.
    React was first deployed on Facebook's newsfeed in 2011 and later on Instagram.com in 2012.

    \noindent \textbf{Pros:}

    React is maintained by Facebook, which has the same benefits as Angular's backing by Google.
    The framework will be actively developed by many people, the documentation will be good, and the product reliable.
    Since React was built more recently than some of the older frameworks, it has some respectable benchmarks in terms of speed.\cite{SpeedReport}
    React can be used to build native apps as well, with the use of the framework React Native, web code can be used to write an app on iOS and Android.

    \noindent \textbf{Cons:}

    React has a fairly big learning curve and has a rather verbose syntax.
    React uses a system where the HTML is embedded in the JavaScript code.
    This makes development for people that don't already know the framework more difficult.
    Any language that is more verbose gives a greater potential for mistakes to be made.
    React is not a full framework.
    There are some features available in other frameworks like router or model management libraries that are not in React.
    A developer needs to be good at making decisions about what kind of frameworks should be added onto React to be able to do everything that other frameworks can do.


    \subsubsection{Vue}

    \textbf{About:}

    Vue is a progressive framework for building user interfaces.
    Vue is designed from the ground up to be incrementally adoptable.
    The core library is focused on the view layer only, and is easy to pick up and integrate with other libraries or existing projects.


    \textbf{Pros:}

    Vue is small at only half the size of Angular and React when running in a production environment.
    Vue is fast with speed benchmarks for simple tasks faster than almost every framework in almost every way measured.
    \footnote{This is not conclusive results that it the fastest out there, but its not something to dismiss.\cite{SpeedReport}}
    Vue is not very opinionated and allows flexibly in development.\cite{Vue}
    Because of its flexibility, it is easy to embed code in existing websites.
    Documentation for Vue is very good, as there is a dedicated core team working on making use of the framework easy and accessible.
    As such, the use of Vue has been trending up at a tremendous rate with trends in searches for vue.js is almost surpassing that of React\cite{vueVSreactSearches}

    \textbf{Cons:}

    Vue being a flexible framework allows programmers to make more mistakes and stylistic decisions that could make it more difficult for other developers to work on.


  \subsection{Conclusion}
    Vue.js is a relatively new framework, but its popularity is still growing, the amount of features it offers is impressive and the size of the package is a major bonus.
    With less code comes less bugs.
    For the scope of this project, having a framework that is lightweight, easy to use, and has all the benefits of the bigger frameworks like Angular and React is a good choice.
    Its possible that the features that this framework offers might not be a requirement for the client, but if a JavaScript framework will help the project, then Vue.js is the best choice for it.

		\section{Hosting}
		    \subsection{Introduction}
				Ninkasi needs to host it's data somehwere, either remotely using cloud hosting or locally using a server of some type.
				Below are the outlined options for small local servers or cloud hosting.
				Space available, scalability, cost, and complexity of each are explained.
				For a cheap, efficient, small server, our team has chosen to work with Raspberry Pi's.
				\subsection{Criteria}
				\begin{itemize}
		  \item Scalable in terms of space, or if not scalable, ability to upgrade or cluster
		  \item Low cost in the long run
		  \item Easy for client to use after we leave this project
		\end{itemize}

      \subsection{Technology}
  			\subsubsection{Intel NUC}
		            The first option for hosting our data is the Intel NUC, or “Next Unit of Computing”.
								It has a low power consumption, idling at 8 watts and is barely audible (alphr). For the size, 5\textquotedblleft square by 1.5\textquotedblleft tall, the NUC is powerful \cite{IntelNUCReview}.
								Each version averages around 32 GB \cite{Intel}, but this is a fixed-size memory space.
								For a small company like Ninkasi, this would work well for the amount of data they are storing.
								If they wanted a scalable product in case their data became larger than 32 GB, that’s where NUC would fall short.
								Price wise, NUC's  fall between 274 and 500 dollars\cite{PCWorld}.
								There are also 4x4\textquotedblleft NUC boards that look very similar to Raspberry Pi's.
								These have 4-32 GB and cost from 120-575 dollars \cite{Intel}.
						\\ \\
						\textbf{Pros: Small and powerful for its size, quiet, good for a small server for a small business.}
						\\
		        \textbf{Cons: Expensive, storage size limitations so not scalable.}

  			\subsubsection{Raspberry Pi}
					Raspberry Pi's are the economically friendly option for hosting.
					The devices run between 5 and 35 dollars per unit and, although it is not the most common use of the technology, can be used as a small server\cite{CopaHost}.
					The raspberry pi itself only contains 1GB of space, and relies of a microSD card (which can hold upto 32GB and still be compatible with the device)\cite{CopaHost}.
					Raspberry Pi’s reliance on the SD card for I/O means the cost of using this technology is not limited to the board.
					This also requires setup that may be confusing for a non-technical user.
					On most version of raspberry pi boards there are 4 USB 2 ports, and 4 Pole stereo/composite video ports\cite{RaspberryPi}, so the device is very customizable.
					For small projects this is a perfect device, but would be a poor decision if the project increases dramatically in size.
					\\\\
					\textbf{Pros: Cheap and you get a lot for your money, server is on site.}
					\\
					\textbf{Cons: Cheap and not reliable, not enough storage space, setting it up and maintaining the hardware would be hard for someone with no little knowledge.}

  			\subsubsection{Cloud Hosting}
		        Cloud hosting is a way of hosting your data on a remote server through a company who charges you for that space.
						People with limited technical knowledge can use this service easily because you can be provided with a software environment that requires no setup or hardware\cite{InterRoute}.
						The system is very scalable, you pay for what you need and if you need more space you can buy more\cite{InterRoute}.
						Unlike the NUC or Raspberry Pi, this method of hosting isn't a one-time payment, you have to pay for the space you reserved each month\cite{TheBalance}.
						Some of the dangers of using cloud are that you are stuck as a “forever” customer if you can't convert your data from the host's system to another system\cite{TheBalance}.
						\\ \\
						\textbf{Pros: Reliable, scalable, easy to use with no technological knowledge, no hardware required.}
						\\
						\textbf{Cons: Have to pay each month, could potentially be stuck with a service if there is not a system for converting your data to anther service/system, poor customer service}


					\subsection{Discussion}
						Raspberry Pi and NUC offer a lot of similar benefits for a small at-home server, but Raspberry Pi’s are much less expensive.
						NUC holds a similar amount of data as a Raspberry Pi combined with an SD card and is less complicated to set up and maintain.
						Cloud hosting is the most appealing for ease of use for our client, but as a monthly service the service provider will charge Ninkasi every month.
						Again, as Ninkasi is a successful company, this is not the most important consideration, but it is costlier in the long run.

						\subsection{Conclusion}
						For this project we will use cloud hosting because our client has specifically requested it.
            Raspberry Pi and NUC both have the benifit of a one time payment, but are more difficult for our client to maintain.
            They also are not scalable if the database were to expand as Ninkasi develops this data entry system.
						Cloud hosting is relatively inexpensive, is easy to use for our nontechnical client, and is scalable with the growth of the company.

		\section{Platform}
		\subsection{Introduction}
		The platform for our web servers will dictate the ease and speed of running concurrent processes, and what kind of stack we will be able to use.
		This section outlines the pros and cons of Nginx, Apache, and Express.js.
		The web servers we will use for our project are Apache, because they are fast, reliable, well documented and supported, and allow us to use Django and Python and relational databases.
		\subsection{Criteria}
		\begin{itemize}
		\item Allows for a relational database
		\item Easy to use and reliable
		\item Well document and supported
		\end{itemize}

      \subsection{Technology}
  			\subsubsection{Nginx}
				NGINX is an open source server software that handles web serving as well as media streaming, caching, load balancing, and more\cite{NGINX}.
				NGINX is known for being very fast, especially around media streaming and serving static content\cite{NGINX}.
				NGINX is 2.5 times faster than Apache when serving static content and consumes less memory of the two when running the same amount of concurrent processes\cite{HostingAd}.
				If Ninkasi were serving a lot of static content and media streams, NGINX would be better as it is much faster for static content than Apache.
				Ninkasi just needs a simple database, so this is not relevant to our project.
				For our project, the advanced features of NGINX are appealing.
				\\ \\
				\textbf{Pros: Good security, faster and less space consuming for static content, many advanced features, handles load balancing well}
				\\
				\textbf{Cons: Most modules don't support dynamic loading, not good OS support for Windows}

  			\subsubsection{Apache}
				Apache is part of the LAMP stack (Linux or other OS, Apache, MySQL, Php or similar lanugage) which is a traditional stack model\cite{UpWork}.
				It has 3 ways of request processing which scale differently.
				Apache has great OS support (better than NGINX which lacks good Windows support) and can run on many systems\cite{HostingAd}.
				It has good security and great documentation and supports dynamic loading well\cite{HostingAd}.
				\\ \\
				\textbf{Pros: Request processing 3 ways, handles load balancing well, excellent documentation, good security, supports dynamic laoding}
				\\
				\textbf{Cons: Slow for static content, doesn't scale well}


  			\subsubsection{Express}
				Express.js is part of the MEAN stack (mongo-db, express.js, angular, node.js) stack and is a great choice for a modern platform if you want language uniformity and the ability to be mobile friendly\cite{UpWork}.
				Since we are not creating apps, the mobile ability is less important but the language uniformity would be helpful.
				That way regardless of what part of the stack the group is working on, they will know the language for all other sides of the stack.
				This system is also very flexible and offers lots of packages and plugins.
				It is an opensource product with a large community and lots of support\cite{JSSolutionsDev}.
				Express is an event  driven server, meaning it has a single threaded framework\cite{JSSolutionsDev}.
				Inexperienced developers may find this confusing if they aren't familiar with the callback nature of this type of server\cite{JSSolutionsDev}.
				Express works with Mongo-db which is a non-relational database. Express doesn’t allow for relational databases, which we need in our project.
				\\ \\
				\textbf{Pros: Flexible, modern, mobile friendly, uniform lanuage for full stack}
				\\
				\textbf{Cons: Confusing if unfamiliar with callback nature of server}

				\subsection{Discussion}
				Our main interests in choosing server software are that it is compatible with a relational database, is easy to use and is well documented.
				NGINX and Apache allow us to use a relational database, while Express does not.
				Express is part of the MEAN stack, which means it’s language compatible and allows JavaScript, but we are limited in flexibility and have to use a NoSQL database.
				NGINX is faster than Apache while serving static content and also consumes less memory. That being said, the PHP runtime between the two types of server software are very similar\cite{HostingAd}.
				Since our website will not have a lot of static content and won’t be streaming any videos or audio, streaming speed doesn’t play a role in our decision.
				\subsection{Conclusion}
				Since Express doesn’t allow for a relational database, it will not be our server software.
				NGINX is faster than Apache, but we won’t be streaming media so this isn’t our main concern.
				Both NGINX and Apache are well documented, well supported, and easy to use.
				Apache has better support for Windows computers and many of Ninkasi’s computers are Windows, so this will be our choice.

		\section{Database Interaction Language}
		\subsection{Introduction}
			The choice of which language with which we will use to interact with our database is an important decision.
			We want our project to be very maintainable, both in the language's support and community, and the code we create with the language.
			For our project we don’t want to work with Php, because of previous experience in other areas and ease of use.
			We also want to use Apache so we can have a relational database and support for Windows OS.
			Because of these wants, we have chosen to use Django as our database interaction language.
			\subsection{Criteria}
			\begin{itemize}
			\item Works with the type of LAMP stack we are looking for
			\item Easy to code with
			\item Language that is popular and continually developed
			\end{itemize}

      \subsection{Technology}
  			\subsubsection{Ruby On Rails}
				Ruby on Rails is a flexible and expressive language that has been at the foreground of database interaction languages in recent years.
				There are a lot of protocols for how to implement web features in Rails.
				Rails is opinionated, meaning most features have a standard way of being create\cite{Medium}.
				This has positives and negatives; making implementing forms or buttons standardized makes Rails easily maintainable, but also doesn't allow the coder as much flexibility in designing their own functions.
				Problems arise when Rails has no opinion on how to implement a web feature, then the style of \textquotedblleft one size fits all\textquotedblright in Rails doesn't work well\cite{Medium}.
				Rails is also very slow in comparison with Node or Php\cite{Medium}.
				For our purposes, Rails would be great for developing our project.
				Rails operates as part of LAMP stacks, which we need in order to implement a relational database.
				\\ \\
				\textbf{Pros: Maintainable, clean, protocols for everything}
				\\
				\textbf{Cons: A lot of abstraction (possibly too much depending on your purposes), not a defined way to do everything, way slow in comparison with other languages.}

  			\subsubsection{Node.js}
					Node.js is a language you can use for full stack development\cite{Medium}.
					It allows Javascript to be run on the server side\cite{Medium}.
					Node is easy to learn, but unlike Rails, you have to build everything from the ground up\cite{NetGuru}.
					With little abstraction, you have total control of how features are implemented but development takes much longer.
					Node updates are often not backward compatible.
					Node can handle multiple requests at the same time and is way better at handling concurrent requests than Ruby\cite{Medium}.
					This means the app takes up less RAM and functions faster.
					Node has a high market demand, so for the intents of our project it would be very maintainable when we hand off our project to Ninkasi to expand.
					\\ \\
					\textbf{Pros: You can build a full stack with one language, easy to learn, strong community, actively developed language.}
					\\
					\textbf{Cons: You have to build every feature from the ground up, often backward incompatible updates.}

  			\subsubsection{Php}
			Php has the benefit of being around for a long time\cite{InfoWorldPhp}.
			This language has a lot of longevity and is constantly updated. It has a large open source community and lots of support, so maintainability would be high\cite{Medium}.
			The language is inconsistent and clunky, so this would diminish our development ease.
			There is a lack of abstraction, similar to Node.js, and often has repetitive code\cite{RailsApps}, something Rails would reduce.
			Php fits into the LAMP stack so we would be able to work with a relational database.
				\\ \\
				\textbf{Pros: Huge ecosystem and large open source community, always being updated, commonly used in industry for a long time.}
				\\
				\textbf{Cons: Very old, language is inconsistent, clunky, lack of abstraction.}

  		\subsubsection{Django}
				Django is a python web framework that works similar to a LAMP framework.
				This stack is comprised of MySQL relational databases, Apache, and python\cite{Bitnami}.
				This is perfect for our need for relational database support. It allows for python front end development, which Connor is interested in.
				Similar to Node, Django is opinionated, having a self-proclaimed “batteries included” philosophy\cite{FullStackPython}.
				This means there is common functionality for building web apps. Authentication, URL routing, database scheme migrations, etc. are all included in Django framework and standardized\cite{FullStackPython}.
				The stability, performance, and community of Django have grown stronger in recent years\cite{FullStackPython}.
				\\ \\
				\textbf{Pros: Works with the stack we want, opinionated lanugage, easy to use, strong stability, performance and community }
				\\
				\textbf{Cons: Not as old as Php or Rails and there potentially less longevity}

				\subsection{Discussion}
				Node.js is a easy to use, well supported language, but it is a part of the MEAN stack so we won’t be able to use it with a relational database.
				Rails and Php are both well documented and have strong open source communities online.
				Php is clunky, but has extreme longevity, so maintainability is high.
				Rails is opinionated, which makes development quick.
				This also makes implementations abstract and farther from the control of the developer, but this is standard in modern web development, so doesn't play a large role in our decision.
        Django is similarly structured, and is great if you want to use Python throughout the whole stack, but is newer than Rails and has a smaller community.

				\subsection{Conclusion}
				Node.js is a easy to use, well supported language, but it is a part of the MEAN stack so we won’t be able to use it with a relational database.
				Php is clunky and inconsistent, so we do not want to develop in Php when we can create similar results with Rails.
        Django is a good option for us, but because of our groups experience with Rails, that will be our choice.
				Rails is a modern, easy to use, abstract language that works with our LAMP-like stack.

		\clearpage

		\section{System Peripherals}
		    \subsection{Introduction}
		        This is an overview of the various peripheral technologies our team may utilize to interface with the data management system.
		        We will analyze the positive and negative impacts of creating a native mobile application, a native desktop application, or a web application.
		        There are benefits and costs associated with each technology as described below.

      \subsection{Technology}
  			\subsubsection{Native Mobile Application}
		        \textbf{Summary:}
		            Native mobile apps play an integral part in many business product lines.
		            These applications are written in languages specific to devices they run on.
		            These are generally recognized as good for sale to external customers\cite{SearchCloudOverview}.

		        \noindent \textbf{Pros:}
		            Native mobile applications have access to all capabilities of the smart phones on which they run.
		            This includes push notifications which may be required in the data management system.
		            Push notifications would allow brewers to receive updates on time-critical tasks without having to refresh the application.
		            Since native mobile applications run on the specific device, this allows them to be used while not connected to the internet.
		            It is unknown whether or not brewers will have good internet connection where they plan to use the application.
		            Another benefit of running natively is that data can be stored directly on the device.
		            This would ensure that data is never stored in a single, vulnerable space.
		            Along these lines, running native mobile applications would reduce hosting needs to only data storage.
		            When developing native mobile applications, most development kits come with "drag-and-drop" interface builders.
		            This feature could reduce development time and increase maintainability once BrewHops team finishes.
		            Finally, native mobile application interfaces run more quickly than mobile-web applications.
		            Depending on how fast-pace the brewers' use of the application will be, this may reduce time spent in application.

		        \noindent \textbf{Cons:}
		            Native mobile applications can drastically increase the development timeline.
		            Each device the application will run on must have its own version of the application.
		            If Ninkasi uses multiple smartphones (E.G iPhone and Android), then each platform requires the application to be written in its native language.
		            Learning new languages and platform specific standards will increase the time to project completion.
		            After creation, deploying native mobile applications is more difficult than web-based applications.
		            Native applications require download from the platform's app store or direct deployment from the developer.
		            This may make it difficult to onboard new brewers or remove brewer's access.
		            Once downloaded, any updates to the interface must be downloaded onto each device.
		            Again, each update must be written for each hardware platform.
		            Finally, if a brewer's device is lost or stolen then there is a potential security risk if Ninkasi data is stored on the device.


  			\subsubsection{Naive Desktop Application}
		        \textbf{Summary:}
		            Many of the same costs and benefits that arise with native mobile applications, come about with native desktop applications.

		        \noindent \textbf{Pros:}
		            When developing native desktop applications, applications have direct access to the host utilities.
		            This is beneficial when many of these functions are needed in the application.
		            Native desktop applications do not require internet connection to function.
		            Depending on how information is stored, internet may still be required to access data.
		            Since the application is running natively, data can be stored in the computer's memory.
		            This can help create a fluid user experience.

		        \noindent \textbf{Cons:}
		            Similar to native mobile applications, native desktop applications require additional versions of the application written for each hardware platform supported.
		            This may not be an issue if Ninkasi uses a single technology ecosystem; but it can make it difficult to switch platforms if there is ever a need.
		            Native desktop applications must be downloaded onto each desktop computer that needs to run the program.
		            This can increase the time needed to onboard new employees who require access to the software.
		            Along with downloading the application onto each device, it'll become necessary to remove the application and all application data from the device when Ninkasi no longer uses the computer.


  			\subsubsection{Web Application}
		        \textbf{Summary:}
		            Web applications are generally recognized as good for cross-platform internal usage\cite{SearchCloudOverview}.

		        \noindent \textbf{Pros:}
		            Web applications allow for each cross-platform functionality.
		            Ninkasi brewers will access the system from their smartphones, desktops, and tablets.
		            A web application will allow us to develop one application that exposes certain functionality based on the device used.
		            These benefits also carry onward to different technology ecosystems.
		            A single web application will run the same on an OSX device as it runs on a Windows device.
		            When deploying web applications, nothing is downloaded since it runs in the browser.
		            Web applications are also easily maintained and updated.
		            They are written using common languages that every developer and many non-developers have seen before.
		            It is possible to create push notifications in web sites\cite{GooglePushNotifications}.

		        \noindent \textbf{Cons:}
		            Unfortunately, since the application is hosted, brewers will need to maintain an internet connection to use the application.
		            Without internet, there will be no way to work with live data from the database.
		            Also, without native software, the application will not have access to hardware specific functions.
		            Web applications also require static storage of the system along with database storage.
		            Depending on how often individuals access the application, this may increase storage and processing requirements.
		            Finally, web application interfaces tend to be slower than native applications\cite{LifeWireOverview}.

		    \subsection{Conclusion}
		    To access and work with the data management system, there are several requirements. Listed below are the criteria along with an evaluation of each technology's ability to meet the requirement:\\


		        \begin{center}
		            \begin{tabular}{ |m{14em}|m{9em}|m{9em}|m{9em}|}
		                \hline
		                    & Native Mobile & Native Desktop & Mobile Web \\

		                \hline
		                    Accessable in Ninkasi cellar & X & & X \\

		                \hline
		                    Accessible in Ninkasi offices & X & X & X \\

		                \hline
		                    Cross Platform &  &  & X \\

		                \hline
		                    Immediately deployable &  &  & X \\

		                \hline
		                    Prior development experience &  &  & X \\

		                \hline
		                     Database connectable & X & X & X \\

		                \hline
		                    Open source & & & X \\

		                \hline
		                    Strong community & X & X & X \\

		                \hline

		            \end{tabular}
		        \end{center}

		    Based on the criteria for our system peripherals, we have decided to build a web application to access the data management system.

		\section{Data Storage}
		    \subsection{Introduction}
		    This is an overview of various types of databases. Databases will be used to store and access information that Ninkasi measures inside of each brewing vat.

      \subsection{Technology}
  			\subsubsection{SQL Database}
		        \textbf{Summary:}
		            SQL Databases, also known as "relational" databases, are widely used to store highly \textit{structured} information\footnote{"Structured" information is data whose structure rarely, if ever, changes and is consistent between entities.}.

		        \noindent \textbf{Pros:}
		            Using SQL databases, it is easy to retrieve necessary information.
		            SQL databases can use transactions which ensure that interactions with the database either complete fully or any changes are rolled back.
		            SQL databases are structured.
		            There exists many methodologies for normalization data to ensure it is stored efficiently\cite{TechwallaSQL}.
		            SQL databases allow for multiple users to access information at the same time\cite{TechwallaSQLPros}.
		            Database administrators can easily control operation permissions for each user in the system.
		            Along with this, many tools exist to help developers optimize queries to access the database.

		        \noindent \textbf{Cons:}
		            Enterprise level SQL servers can be expensive to develop and maintain.
		            Depending on the database, tables can have limits on their structure\cite{TechwallaSQLCons}.


  			\subsubsection{NoSQL Database}
		        \textbf{Summary:}
		            NoSQL databases, also known as "non-relational" databases, are widely used to increase speed and decrease storage space of \textit{unstructured} information\footnote{"Unstructured" data is information that is not consistent between entities.}.

		        \noindent \textbf{Pros:}
		            There is no defined structure to the database.
		            The structure is dynamic and can be updated without having to shutdown the database\cite{MongoDBProsCons}.
		            Another benefit of NoSQL databases is their ability to handle "Big Data".
		            There are tools available such as Hadoop that can handle the processing requirements of big data systems\cite{NoSQLProsCons}.
		            Ninkasi produces high amounts of information that is monitored daily.
		            Ideally, the client would prefer to track this information more frequently, which would lead to larger amounts of information.
		            Also, "NoSQL databases typically use clusters of cheap commodity servers to manage the exploding data and transaction volumes."\cite{NoSQLProsCons}.
		            This may be essential if the BrewHops team uses Raspberry Pis or Intel Nucs to host system information.


		        \noindent \textbf{Cons:}
		            While there are benefits to using a NoSQL database, there are also drawbacks.
		            NoSQL technologies are typically open-source projects that do not have the support and resources that major relational database management systems have\cite{NoSQLProsCons}.
		            There are still several security issues associated with NoSQL databases\cite{NoSQLSecurityIssues}.
		            There is no guarantee data consistency when using NoSQL databases\cite{ChannelFutures}.
		            They use "eventual consistency" which implies times of inconsistency.
		            Some NoSQL databases do not automatically shard data to properly spread data across clusters\cite{ChannelFutures}.
		            This will only be an issue if Ninkasi data needs will require multiple nodes.


  			\subsubsection{Flat File Database}
		        \textbf{Summary:}
		            Flat file databases are commonly used to store configuration settings for software systems.

		        \noindent \textbf{Pros:}
		            Incredibly simple to set up, literally just a file.
		            Easily read since there is a standard structure using delimeters or line length.

		        \noindent \textbf{Cons:}
		            Duplicate data, any new entries require a new row
		            No transactions
		            Not usually accessed via network since they're offline entities that are part of the operating system\cite{Techwalla}.
		            "... a flat file database is disadvantageous to a network user, who is accessing a multi-access, multi-tasking relational online databse which can be viewed from many different aspects." \cite{Techwalla}.
		            "... not encountered as 'databases', but more as configuration files..." \cite{Techwalla}.

		    \subsection{Conclusion}
		        Listed below are the requirements of the database system that will be used in this data management system.

		        \begin{center}
		            \begin{tabular}{| m{15em} | m{10em} | m{10em} | m{10em} |}
		                \hline
		                    & SQL Database & NoSQL Database & Flat Files \\

		                \hline
		                    Minimal data duplication & X & X & \\

		                \hline
		                    Data Integrity & X &  &  \\

		                \hline
		                    Data retrieval reliability & X &  & X \\

		                \hline
		                    Strong Documentation & X & X & X \\

		                \hline
		                    Free to use & X\footnote{There are certain versions of SQL databases that are free.} & X & X \\

		                \hline
		            \end{tabular}
		        \end{center}

		        Based on the criteria, we will store our data in a relational SQL database.

		\section{Task Runners and Module Bundlers}
		    \subsection{Introduction}
		        This section gives an overview on various task runners and module bundlers.
		        It is common practice in web development to use small open source packages/libraries to complete specific tasks.
		        Each of these libraries comes with dependencies along with other tasks that need to be handled before deployment.
		        Task runners and module bundlers take care of these dependencies and performing routine pre-deployment operations.

      \subsection{Technology}
  			\subsubsection{Grunt}
		        \textbf{Summary:}
		            Grunt is a tool that allows "front-end developers to run predefined reptitive tasks"\cite{TaskRunners}. This is commonly used in enterprise level web applications.

		        \noindent \textbf{Pros:}
		            Grunt is built on top of NodeJS which makes it integrate easily with other Node based applications.
		            Grunt is the oldest tool of its kind and has a strong community behind it.
		            This also ensures the stability and further development of Grunt.
		            Along with this, Grunt has thousands of modules available for developers to use.
		            An important difference between Grunt and other task runners is its focus on configuration over code.
		            To make Grunt work for you, you configure the tasks it completes.

		        \noindent \textbf{Cons:}
		            There are some downsides to Grunt as well.
		            Developers reuse configuration objects to automate their tasks.
		            This can lead to bloated Grunt scripts that are not maintainable\cite{TaskRunners}.

  			\subsubsection{Gulp}
		        \textbf{Summary:}
		            Gulp is a task runner that requires developers to write code to automate repetitive tasks\cite{TaskRunners}.
		            Gulp was developed after Grunt.

		        \noindent \textbf{Pros:}
		            A major benefit of Gulp is its use of Javascript functions to define when and how tasks should execute.
		            Another benefit is that Gulp scripts tend to run more quickly than Grunt configuration\cite{GulpGruntSpeed}.
		            This is possible ecause of Gulp's "stream usage and in-memory operations"\cite{TaskRunners}.

		        \noindent \textbf{Cons:}
		            Gulp relies on several paradigms that are not common practice in other areas of development.
		            These paradigms can make it difficult for new developers to create efficently create automation scripts.
		            There is no reason to use a task runner on small projects if the developer can quickly and easily perform the tasks by hand.

  			\subsubsection{WebPack}
		        \textbf{Summary:}
		            WebPack differs from Grunt and Gulp in that it is a module bundler.
		            It takes various modules that all have dependencies and bundles them together in a way that reduces redundancy\cite{WebPack}.
		            It creates a dependency graph of all modules needed in your system and consolidates them into individual bundles\cite{WebPack}.

		        \noindent \textbf{Pros:}
		            The major benefit of using WebPack is its ability to reduce software size through its consolidation method of individual modules.
		            Another benefit of WebPack is using hot-reloading to more efficiently develop the frontend of an application\cite{TaskRunners}.
                The biggest pro is that Vue.js has a command line interface designed to ease development with Vue components.
                This command line interface uses Webpack as the task runner.

		        \noindent \textbf{Cons:}
		            Finally, the main difficulty with WebPack is it can be difficult to configure at first and there exists a learning curve to its technology\cite{TaskRunners}.
		            WebPack is a technology that is extremely useful for major enterprise level applications, but tends to be over-powered for small applications.

		    \subsection{Conclusion}
		        Listed below are the various criteria needed in a task runner and module bundling system.

		    \begin{center}
		            \begin{tabular}{| m{15em} | m{10em} | m{10em} | m{10em} |}
		                \hline
		                    & Grunt & Gulp & WebPack \\

		                \hline
		                    Software size reduction & X & X & X \\

		                \hline
		                    Automatable for deployment & X & X & X \\

		                \hline
		                    Easy to learn &  & X &  \\

		                \hline
		                    Strong community & X & X & X \\

		                \hline
		                    Strong Documentation & X & X & X \\

		                \hline
		                    Free to use & X & X & X \\

		                \hline
                        Integrates well with Vue.js &  &  & X \\

                    \hline
		            \end{tabular}
		        \end{center}

		        Based on the criteria, we will be using Webpack to automate repetitive tasks.

					\clearpage

\section{Appendix}

  \begin{itemize}
    \item Macros - Segments of code that can be called multiple times and get compiled in. This
    \item CSS: Cascading Style Sheets - A language that defines the style of a websites interface.
    \item W3C: World Wide Web Consortium - A committee in charge of deciding web standards. They vote yes or no on implementing new standards, and then any web browser that wants to remain competitive should implement those features as soon as they can.
    \item Compiled/Compiler - A compiler is a program that takes a programming language and converts it into another programming language. Generally done because the first language is easy to write and the second language can be run on the computer.
    \item Model-View-Controller - A software architectural pattern for implementing user interfaces on computers. It divides a given application into three interconnected parts. This is done to separate internal representations of information from the ways information is presented to, and accepted from, the user.
  \end{itemize}

\begin{landscape}
  \begin{table}[]
  \centering
  \caption{Charts - D3.js vs Chart.js vs C3.js}
  \label{my-label}
    \begin{tabular}{lllll}
      & \lstinputlisting{d3.js} & \lstinputlisting{chart.js} & \lstinputlisting{c3.js} & \\
    \end{tabular}
  \end{table}
\end{landscape}

\clearpage

\bibliography{references}{}
\bibliographystyle{IEEEtran}

\end{document}
