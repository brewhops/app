% Technology Review
% CS 461 - CS Senior Capstone
% Fall 2017
% Authors: Connor Christensen

% Paper requirements:
% 1. Your team number, your name, and project name
% 2. Your role in the project
% 3. What you are trying to accomplish (at a very high level) -- use info from your problem statement, but focus on the specific sub-pieces you, as an individual, are working on.
% 4. Three possible technologies that could be used to accomplish the different pieces you selected to examine in more detail. Identify these potential technologies even if your client has told you which one to use.
% 5. After conducting research and analyzing trade-offs, identify which technology you have selected for each piece of the project, and why. Convince the reader your analysis is unbiased and well-considered.


\documentclass[draftclsnofoot,onecolumn,letterpaper,10pt,compsoc]{IEEEtran}

% Packaging
\usepackage{geometry}
\usepackage{hyperref}
\usepackage{titling}
\usepackage{color}
\usepackage{listings}
\usepackage{cite}
\usepackage{pdfpages}
\usepackage{pdflscape}
\usepackage{url}
\usepackage{xargs}                      % Use more than one optional parameter in a new commands


\usepackage[pdftex,dvipsnames]{xcolor}  % Coloured text etc.
\usepackage[colorinlistoftodos,prependcaption,textsize=tiny]{todonotes}
\newcommandx{\note}[2][1=]{\todo[linecolor=red,backgroundcolor=red!25,bordercolor=red,#1]{#2}}

\newcommand\question[1]{\footnote{\textcolor{red}{#1}}}


\definecolor{lightgray}{rgb}{.9,.9,.9}
\definecolor{darkgray}{rgb}{.4,.4,.4}
\definecolor{purple}{rgb}{0.65, 0.12, 0.82}

\lstdefinelanguage{JavaScript}{
  keywords={typeof, new, true, false, catch, function, return, null, catch, switch, var, if, in, while, do, else, case, break},
  keywordstyle=\color{blue}\bfseries,
  ndkeywords={class, export, boolean, throw, implements, import, this},
  ndkeywordstyle=\color{darkgray}\bfseries,
  identifierstyle=\color{black},
  sensitive=false,
  comment=[l]{//},
  morecomment=[s]{/*}{*/},
  commentstyle=\color{purple}\ttfamily,
  stringstyle=\color{red}\ttfamily,
  morestring=[b]',
  morestring=[b]"
}

\lstset{
   language=JavaScript,
   extendedchars=true,
   basicstyle=\footnotesize\ttfamily,
   showstringspaces=false,
   showspaces=false,
   tabsize=2,
   breaklines=true,
   showtabs=false,
   captionpos=b,
   basicstyle=\tiny
}



% Paper type
\geometry{letterpaper, margin=.75in}

% Title page
\title{CS 461 - CS Senior Capstone
  \\Team 10 - Brewing Operations Management System
	\\Fall 2017
	\\Technology Review
}


\author{
	Connor I. Christensen \\
	\texttt{chriconn@oregonstate.edu}
}

\begin{document}

\begin{titlingpage}
    \maketitle
    \begin{abstract}
      Ninkasi Brewing Company is based in Eugene, Oregon, producing and distributing tens of thousands of barrels of beer each year across the United States and Canada.
      Maintaining a consistently high quality product is vital to the longevity of Ninkasi.
      Scalable brewing is a detail-oriented and organized process; companies require unique data tracking methods to fit the needs of their brewing process.
      Ninkasi currently tracks brewery data using Microsoft Excel spreadsheets.
      This process is laborious, time consuming, and error prone.
      The goal of this project is to create a brewing operations management system to add and monitor brewery data.
      The brewing and cellar team members will utilize the system to access up-to-date information.
      \\
      \textbf{Keywords:} Brewing, Operations, Management
    \end{abstract}
		\pagebreak
		\tableofcontents
\end{titlingpage}

\section{Introduction}


\section{Data Visualization}
  \subsection{Introduction}
  This is an overview of libraries written for data display in web browsers.
  Each of these libraries are build out of the web language JavaScript to allow for live interaction with the user.
  Table 1 shows the code needed to set up a basic bar graph using each library.
  These yield approximately the same visual results.
  All have the con that they require JavaScript to be enabled.
  All of them are open source and free.

  \subsection{D3.js}
  D3 stands for Data-Driven Documents and it is a powerhouse of a library for data visualization.
  It was created by a team of PhD graduates working out of the Stanford Visualization Group and is essentially the library for creating amazing data visualization.
  It has a relatively bulky setup, but allows fine grained detail and control over how data is displayed.
  This library is so powerful that it functions more as a library for painting than for graphing.
  D3 uses SVG's \footnote{\textbf{S}calable \textbf{V}ector \textbf{G}raphics: A visual component defined mathematically. It can scale up or down to any size without loss of image quality.} which allows it to be able to create any 2d shape that can be mathematically defined.
  SVG's do not require JavaScript to work, though D3 can't run without JavaScript enabled.
  The D3 library uses pre-build JavaScript functions to select elements, create SVG objects, and be able to preform a wide number of transformations on them.
  Transformations are just as precise as the drawing abilities and are limited only by the processing power of the computer and the skill of the programmer.
  It uses jQuery and CSS styled selection and modification of content for flexibility and ease of use.
  Objects that are made with D3 are easily syllable with CSS, meaning that data display can inherit styling rules to maintain a consistent look across the platform.

  D3 has a large community

  There are a huge number of online examples

  It is backwards compatible

  Selections are hard and the learning curve is steep

  \subsection{C3.js}
  C3 is a package built off D3 and is formatted specifically for creating graphs.
  Since it is built off D3, C3 requires D3 to be installed, which causes many lines of code to be added to the project.
  There are nearly 10,000 lines of JavaScript in both D3 and C3, bringing in almost 20,000 lines of code for data display.
  The interface is much easier to use, the tradeoff being more limiting in its expressions.

  It has a good set of examples

  It can easily switch between chart types

  Ledgend comes out of the box

  \subsection{Chart.js}
  Chart.js is a JavaScript library that allows you to draw different types of charts using the HTML5 canvas element
  \footnote{HTML Canvas: A new web standard that allows web programmers to create computer graphics created and rendered in the browser.}.
  Use of the canvas comes with a few drawbacks, the most common issue being that it cannot scale without loss of quality.
  The HTML5 Canvas specification recommends that authors should not use the canvas element when they have other more suitable means available.\cite{CanvasVsSVG}
  Canvas is good for 3d graphics, but this ability is not beneficial if you simply want a bar graph.
  If you are drawing little details all very close together, canvas is great for that.
  Canvas is not very accessible as it is just drawing pixels and no data can be extracted by assistive technology or bots.

  Chart.js is very responsive and the documentation is very good.

  It can only draw 6 graph types.

  it is lightweight.

  \subsection{Conclusion}


  \begin{landscape}
    \begin{table}[]
    \centering
    \caption{Charts - D3.js vs Chart.js vs C3.js}
    \label{my-label}
      \begin{tabular}{lllll}
        & \lstinputlisting{d3.js} & \lstinputlisting{chart.js} & \lstinputlisting{c3.js} & \\
      \end{tabular}
    \end{table}
  \end{landscape}


\section{Styling}
  \subsection{Introduction}
	\subsection{Preprocessors}

  Pros:
  \begin{itemize}
    \item Lightweight
    \item Easy on developers
    \begin{itemize}
      \item Modular code abilities
      \item Less redundency in code
      \item Really easy to go in and change if you write your code right
      \item Faster development compared to regular CSS
    \end{itemize}
    \item Extras steps are done before deploying
    \item Large ammount of users
    \item Much more freedom for writing code
  \end{itemize}

  Cons:
  \begin{itemize}
    \item Potentially less well known
    \item Compiler could change or dissapear in the future
    \item No out of the box visual features
    \item Potentially harder to maintain
    \item Requires compiler
  \end{itemize}


	\subsection{CSS}
	\subsection{Bootstrap}

  Pros:
  \begin{itemize}
    \item Developed by twitter
    \item Huge ammount of users and documentation
    \item Battle tested and very few bugs
    \item Includes HTML, JS and CSS components built in
    \item Mobile inclusive design
    \item Buttons, menus and icons included
    \item Works well with jQuery
    \item Standardized and many developers know the framework
    \item Open source and free
  \end{itemize}

  Cons:
  \begin{itemize}
    \item Really large ammounts of code
    \item Hard to overwrite elements with custom code
    \item Creates bulky HTML
    \item Javascript tied to jQuery
    \item All Bootstrap websites end up looking the same
    \item Built from LESS
    \item Slows down the site
  \end{itemize}

  \subsection{Conclusion}

\section{Interactive Web Frameworks}
  \subsection{Introduction}
	\subsection{Angular}
	\subsection{React}
  \subsection{Vue}
	\subsection{Ember}
  \subsection{No Framework}
  \subsection{Conclusion}


\bibliography{references}{}
\bibliographystyle{plain}

\end{document}
