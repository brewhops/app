% Technology Review
% CS 461 - CS Senior Capstone
% Fall 2017
% Authors: Connor Christensen, Lily Shellhammer, William Buffum


\documentclass[draftclsnofoot,onecolumn,letterpaper,10pt,compsoc]{IEEEtran}

% Packaging
\usepackage{geometry}
\usepackage{hyperref}
\usepackage{titling}
\usepackage{color}
\usepackage{listings}
\usepackage{cite}
\usepackage{pdfpages}
\usepackage{xargs}                      % Use more than one optional parameter in a new commands
%\usepackage[pdftex,dvipsnames]{xcolor}  % Coloured text etc.
%
%\usepackage[colorinlistoftodos,prependcaption,textsize=tiny]{todonotes}
%\newcommandx{\note}[2][1=]{\todo[linecolor=red,backgroundcolor=red!25,bordercolor=red,#1]{#2}}

%\newcommand\question[1]{\footnote{\textcolor{red}{#1}}}

% Paper type
\geometry{letterpaper, margin=.75in}

% Title page
\title{CS 461 - CS Senior Capstone
	\\Fall 2017
	\\Technology Review
}


\author{
	Connor I. Christensen \\
	\texttt{chriconn@oregonstate.edu}
	\\
	Lily M. Shellhammer \\
	\texttt{shellhal@oregonstate.edu}
	\\
	William B. Buffum \\
	\texttt{buffumw@oregonstate.edu}
}

\begin{document}

\begin{titlingpage}
    \maketitle
    \begin{abstract}
			\\
			\textbf{Keywords:}
    \end{abstract}
		\pagebreak
		\tableofcontents
\end{titlingpage}

\section{Introduction}

\section{System Peripherals}
	\subsection{Native Mobile Application}
    Summary:
    
    Pros:
        Do not need internet connection to view application.
        Have direct access to device capabilities.
        Able to distribute application without giving client the code.
        Development kits generally have "drag-and-drop" interface builders.
        If BrewHops team wanted to continue with this project after             Capstone, this would allow us to distribute via app stores.
    Cons:
        Different languages and development processes for different devices.
        Not maintainable for Ninkasi.
        Must deploy through app store for specific device or given directly     to each device by developer.
        Must download updates after initial release (through methods listed     above).
        If deployed on App stores, must be approved, this can delay             deployment.
    
    Sources:
    https://goo.gl/QkQdvf
    https://www.lifewire.com/native-apps-vs-web-apps-2373133
        
    
	\subsection{Naive Desktop Application}
    Summary:
    
    Pros:
    
    Cons:
    
	\subsection{Web Application}
    Summary:
    
    Pros:
    Easy to maintain (same languages for all devices).
    No approval from app stores.
    
    Cons:

\section{Technology 2}
	\subsection{Technology 2.1}
	\subsection{Technology 2.2}
	\subsection{Technology 2.3}

\section{Technology 3}
	\subsection{Technology 3.1}
	\subsection{Technology 3.2}
	\subsection{Technology 3.3}

\end{document}
