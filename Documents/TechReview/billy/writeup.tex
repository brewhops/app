% Technology Review
% CS 461 - CS Senior Capstone
% Fall 2017
% Authors: William Buffum


\documentclass[draftclsnofoot,onecolumn,letterpaper,10pt,compsoc]{IEEEtran}

% Packaging
\usepackage{geometry}
\usepackage{hyperref}
\usepackage{titling}
\usepackage{color}
\usepackage{listings}
\usepackage{cite}
\usepackage{pdfpages}
\usepackage{xargs}                      % Use more than one optional parameter in a new commands
%\usepackage[pdftex,dvipsnames]{xcolor}  % Coloured text etc.
%
%\usepackage[colorinlistoftodos,prependcaption,textsize=tiny]{todonotes}
%\newcommandx{\note}[2][1=]{\todo[linecolor=red,backgroundcolor=red!25,bordercolor=red,#1]{#2}}

%\newcommand\question[1]{\footnote{\textcolor{red}{#1}}}

% Paper type
\geometry{letterpaper, margin=.75in}

% Title page
\title{CS 461 - CS Senior Capstone
	\\Fall 2017
	\\Technology Review
}


\author{
	William B. Buffum \\
	\texttt{buffumw@oregonstate.edu}
}

\begin{document}

\begin{titlingpage}
    \maketitle
    \begin{abstract}
      Ninkasi Brewing Company is based in Eugene, Oregon, producing and distributing tens of thousands of barrels of beer each year across the United States and Canada.
      Maintaining a consistently high quality product is vital to the longevity of Ninkasi.
      Scalable brewing is a detail-oriented and organized process; companies require unique data tracking methods to fit the needs of their brewing process.
      Ninkasi currently tracks brewery data using Microsoft Excel spreadsheets.
      This process is laborious, time consuming, and error prone.
      The goal of this project is to create a brewing operations management system to add and monitor brewery data.
      The brewing and cellar team members will utilize the system to access up-to-date information.
      \\
      \textbf{Keywords:} Brewing, Operations, Management
    \end{abstract}
		\pagebreak
		\tableofcontents
\end{titlingpage}

\section{Introduction}

\section{System Peripherals}
    \subsection{Introduction}
        This is an overview of the various peripheral technologies our team may utilize to interface with the data management system. 
        We will analyze the positive and negative impacts of creating a native mobile application, a native desktop application, or a web application.
        There are benefits and costs associated with each technology as described below.

	\subsection{Native Mobile Application}
        \subsubsection{Summary}
            Native mobile apps play an integral part in many business product lines.
            These applications are written in languages specific to devices they run on.
            These are generally recognized as good for sale to external customers\cite{SearchCloudOverview}.

        \subsubsection{Pros}
            Native mobile applications have access to all capabilities of the smart phones on which they run. 
            This includes push notifications which may be required in the data management system. 
            Push notifications would allow brewers to receive updates on time-critical tasks without having to refresh the application.
            Since native mobile applications run on the specific device, this allows them to be used while not connected to the internet.
            It is unknown whether or not brewers will have good internet connection where they plan to use the application\question{We need to ask Daniel about internet availability.}.
            Another benefit of running natively is that data can be stored directly on the device.
            This would ensure that data is never stored in a single, vulnerable space.
            Along these lines, running native mobile applications would reduce hosting needs to only data storage.
            When developing native mobile applications, most development kits come with "drag-and-drop" interface builders. 
            This feature could reduce development time and increase maintainability once BrewHops team finishes.
            Finally, native mobile application interfaces run more quickly than mobile-web applications.
            Depending on how fast-pace the brewers' use of the application will be, this may reduce time spent in application.

        \subsubsection{Cons}
            Native mobile applications can drastically increase the development timeline.
            Each device the application will run on must have its own version of the application.
            If Ninkasi uses multiple smartphones (E.G iPhone and Android), then each platform requires the application to be written in its native language.
            Learning new languages and platform specific standards will increase the time to project completion.
            After creation, deploying native mobile applications is more difficult than web-based applications.
            Native applications require download from the platform's app store or direct deployment from the developer.
            This may make it difficult to onboard new brewers or remove brewer's access.
            Once downloaded, any updates to the interface must be downloaded onto each device.
            Again, each update must be written for each hardware platform.
            Finally, if a brewer's device is lost or stolen then there is a potential security risk if Ninkasi data is stored on the device.
    
    
	\subsection{Naive Desktop Application}
        \subsubsection{Summary}
            Many of the same costs and benefits that arise with native mobile applications, come about with native desktop applications.
        
        \subsubsection{Pros}
            When developing native desktop applications, applications have direct access to the host utilities. 
            This is beneficial when many of these functions are needed in the application.
            Native desktop applications do not require internet connection to function.
            Depending on how information is stored, internet may still be required to access data.
            Since the application is running natively, data can be stored in the computer's memory.
            This can help create a fluid user experience.
    
        \subsubsection{Cons}
            Similar to native mobile applications, native desktop applications require additional versions of the application written for each hardware platform supported.
            This may not be an issue if Ninkasi uses a single technology ecosystem; but it can make it difficult to switch platforms if there is ever a need.
            Native desktop applications must be downloaded onto each desktop computer that needs to run the program.
            This can increase the time needed to onboard new employees who require access to the software.
            Along with downloading the application onto each device, it'll become necessary to remove the application and all application data from the device when Ninkasi no longer uses the computer.
    
    
	\subsection{Web Application}
        \subsubsection{Summary}
            Web applications are generally recognized as good for cross-platform internal usage\cite{SearchCloudOverview}.
    
        \subsubsection{Pros}
            Web applications allow for each cross-platform functionality.
            Ninkasi brewers will access the system from their smartphones, desktops, and tablets.
            A web application will allow us to develop one application that exposes certain functionality based on the device used.
            These benefits also carry onward to different technology ecosystems.
            A single web application will run the same on an OSX device as it runs on a Windows device.
            When deploying web applications, nothing is downloaded since it runs in the browser.
            Web applications are also easily maintained and updated.
            They are written using common languages that every developer and many non-developers have seen before.
    
        \subsubsection{Cons}
            Unfortunately, since the application is hosted, brewers will need to maintain an internet connection to use the application.
            Without internet, there will be no way to work with live data from the database.
            Also, without native software, the application will not have access to hardware specific functions such as push notifications.
            There will be no way to notify brewers when certain tasks need to happen immediately.
            Web applications also require static storage of the system along with database storage.
            Depending on how often individuals access the application, this may increase storage and processing requirements.
            Finally, web application interfaces tend to be slower than native applications\cite{LifeWireOverview}.


\section{Technology 2}
	\subsection{Technology 2.1}
	\subsection{Technology 2.2}
	\subsection{Technology 2.3}

\section{Technology 3}
	\subsection{Technology 3.1}
	\subsection{Technology 3.2}
	\subsection{Technology 3.3}

\bibliography{references}{}
\bibliographystyle{plain}

\end{document}
