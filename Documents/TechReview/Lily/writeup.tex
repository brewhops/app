
% Technology Review
% CS 461 - CS Senior Capstone
% Fall 2017
% Author: Lily Shellhammer


\documentclass[draftclsnofoot,onecolumn,letterpaper,10pt,compsoc]{IEEEtran}

% Packaging
\usepackage{geometry}
\usepackage{hyperref}
\usepackage{titling}
\usepackage{color}
\usepackage{listings}
\usepackage{cite}
\usepackage{pdfpages}
\usepackage{xargs}                      % Use more than one optional parameter in a new commands
%\usepackage[pdftex,dvipsnames]{xcolor}  % Coloured text etc.
%
%\usepackage[colorinlistoftodos,prependcaption,textsize=tiny]{todonotes}
%\newcommandx{\note}[2][1=]{\todo[linecolor=red,backgroundcolor=red!25,bordercolor=red,#1]{#2}}

%\newcommand\question[1]{\footnote{\textcolor{red}{#1}}}

% Paper type
\geometry{letterpaper, margin=.75in}

% Title page
\title{CS 461 - CS Senior Capstone
	\\Fall 2017
	\\Technology Review - Group 10
}


\author{
	Lily Shellhammer \\
	\texttt{shellhal@oregonstate.edu}
}

\begin{document}

\begin{titlingpage}
    \maketitle
    \begin{abstract}
      Ninkasi Brewing Company is based in Eugene, Oregon, producing and distributing tens of thousands of barrels of beer each year across the United States and Canada.
      Maintaining a consistently high quality product is vital to the longevity of Ninkasi.
      Scalable brewing is a detail-oriented and organized process; companies require unique data tracking methods to fit the needs of their brewing process.
      Ninkasi currently tracks brewery data using Microsoft Excel spreadsheets.
      This process is laborious, time consuming, and error prone.
      The goal of this project is to create a brewing operations management system to add and monitor brewery data.
      The brewing and cellar team members will utilize the system to access up-to-date information.
      \\
      \textbf{Keywords:} Brewing, Operations, Management
    \end{abstract}
		\pagebreak
		\tableofcontents
\end{titlingpage}

\section{Introduction}
My job in the Ninkasi BrewHops project is to develop and organize the middle of the stack: the languages that connect the database to the front end.
I will be working on the databse interaction languages and how to pull information from/put information into the database.
This documents explains three different technoligies for hosting, platform and database interaction language possibilities.
\\
DISCLAIMER: This is a rough draft and I'm working on the citations. This is not what I think a professional document's citations should look like, don't worry.
\section{Hosting}
    \subsection{Hosting Summary}
		Ninkasi needs to host it's data somehwere, either remotely using cloud hosting or locally using a server of some type.
		Below are the outlined options for small local servers or cloud hosting.
		Through the course of this research, our group has decided to use cloud hosting for its flexibility, ease of use, and scalability.
	\subsection{Intel NUC}
            The first option for hosting our data is the Intel NUC, which stands for Next Unit of Computing.
						For an always-on server that doesn't need to hold a huge amount of data, the NUC is a great option.
						It has a low power consumption, idling at 8 watts and is barely audible (alphr). For the size, 5\" square by 1.5\" tall , the NUC is powerful ...(alphr).
						For a small company like Ninkasi, this would work well for the amount of data they are storing.
						If they wanted a scalable product, that’s where NUC would fall short.
						It has a limited amount of space and has a load capacity....
						Price wise, NUC's fall between 274 and 500 dollars (pcworld).
						There are also 4x4\" NUC boards that look very similar to Raspberry Pi's.
						These have 4-32 GB and cost from 120-575 dollars (intel).
						Raspberry Pi\'s hold similar amounts of data and are much less expensive.
				\\ \\
				\textbf{Pros: Small and powerful for its size, quiet, good for a small server for a small business.}
				\\
        \textbf{Cons: Expensive, storage size limitations so not scalable.}
						\\
        Sources:
        http://www.alphr.com/intel/intel-nuc/32113/intel-nuc-review
        https://www.pcworld.com/article/3039441/hardware/the-mighty-mini-pc-we-review-everything-from-bare-bones-to-ready-made.html
				https://www.intel.com/content/www/us/en/products/boards-kits/nuc/boards.html



	\subsection{Raspberry Pi}
			Raspberry Pi's are the economically friendly option for hosting.
			The devices run between 5 and 35 dollars per unit and can be used as a small server(raspberrypi).
			Space is limited and the devices have poor CPU performance(copahost). They also have poor IO performance as the device relies on an SD card(copahost).
			For small projects this is a perfect device, but for a large craft brewery to rely on for their database, this is too small of an option.
			Ninkasi needs something larger and more suitable for the long term.
			\\ \\
			\textbf{Pros: Cheap and you get a lot for your money, server is on site.}
			\\
			\textbf{Cons: Cheap and not reliable, not enough storage space, setting it up and maintaining the hardware would be hard for someone with no little knowledge.}
			\\
			Sources:
			https://www.raspberrypi.org/ \\
			https://www.copahost.com/blog/is-it-possible-to-run-a-web-server-in-a-raspberry-pi-3-as-a-dedicated-server/ \\ \\

	\subsection{Cloud Hosting}
        Cloud hosting is a way of hosting your data on a remote server.
				People with limited technical knowledge can use this service easily because you can be provided with a software environment that requires no setup or hardware (interoute).
				The system is very scalable, you pay for what you need and if you need more space you can buy more (interroute).
				Unlike the NUC or Raspberry Pi, this method of hosting isn't a one-time payment, you have to pay for the space you reserved each month (thebalance).
				Some of the dangers of using cloud are that you are stuck as a \"forever\" customer if you can't convert your data from the host's system to another system (the balance).
				Cloud is the best choice for this project.
				Without any technical knowledge or hardware knowledge, and not knowing if the scale of the data will increase a lot in the coming years, having the ease and flexibility of cloud hosting would benefit Ninkasi.
				\\ \\
				\textbf{Pros: Reliable, scalable, easy to use with no technological knowledge, no hardware required.}
				\\
				\textbf{Cons: Have to pay each month, could potentially be stuck with a service if there is not a system for converting your data to anther service/system, poor customer service}
				\\
    Sources: \\
        https://www.thebalance.com/disadvantages-of-cloud-computing-4067218 \\
				https://www.interoute.com/what-cloud-hosting \\ \\


\section{Platform}
\subsection{Platform Summary}
The platform for our web servers will dictate the ease and speed of running concurrent processes, and what kind of stack we will be able to sue.
The web servers we will use for our project are ...
	\subsection{Nginx}
		NGINX is 2.5 times faster than Apache when serving static content and consumes less memory of the two when running the same amount of concurrent processes(hostingadvice).
		That being said, the PHP runtime between the two is very similar (hostingadvice).
		If Ninkasi were serving a lot of static content and media streams, NGINX would be better as it is much faster for static content than Apache.
		Ninkasi just needs a simple database, so this is not relevant to our project.
		For our project, the advanced features of NGINX are appealing.
		\\ \\
		\textbf{Pros: Good security, faster and less space consuming for static content, many advanced features, handles load balancing well}
		\\
		\textbf{Cons: Most modules don't support dynamic loading}
		\\
		Sources: \\
		http://www.hostingadvice.com/how-to/nginx-vs-apache/\\ \\
	\subsection{Apache}
		Apache is part of the LAMP stack (Linux or other OS, Apache, MySQL, Php or similar lanugage) which is a traditional stack model (upwork).
		It has 3 ways of request processing which scale differently.
		Apache has great OS support (better than NGINX which lacks good Windows support) and can run on many systems (hostingadvice).
		It has good security and great documentation and supports dynamic loading well (hostingadvice).
		This would be a good choice for this project, but with the language uniformity of Express, isn't our main choice.
		\\ \\
		\textbf{Pros: Request processing 3 ways, handles load balancing well, excellent documentation, good security, supports dynamic laoding}
		\\
		\textbf{Cons: Slow for static content, doesn't scale well}
		\\
		Sources:\\
		http://www.hostingadvice.com/how-to/nginx-vs-apache/\\
		https://www.upwork.com/hiring/development/choosing-the-right-software-stack-for-your-website/ \\ \\
	\subsection{Express}
		Express.js is part of the MEAN stack (mongo-db, express.js, angular, node.js) stack and is a great choice for a modern platform if you want language uniformity and the ability to be mobile friendly(upwork).
		Since we are not creating apps, the mobile ability is less important but the language uniformity would be helpful.
		That way regardless of what part of the stack the group is working on, they will know the language for all other sides of the stack.
		This system is also very flexible and offers lots of packages and plugins.
		It is an opensource product with a large community and lots of support(jssolutionsdev).
		Express is an event  driven server, meaning it has a single threaded framework(jssolutionsdev).
		Inexperienced developers may find this confusing if they aren\'t familiar with the callback nature of this type of server (jssolutionsdev).
		\\ \\
		\textbf{Pros: Flexible, modern, mobile friendly, uniform lanuage for full stack}
		\\
		\textbf{Cons: Confusing if unfamiliar with callback nature of server}
		\\
		Sources:\\
		https://jssolutionsdev.com/blog/express-mobile-app-development/ \\
		https://www.upwork.com/hiring/development/choosing-the-right-software-stack-for-your-website/ \\ \\

\section{Database Interaction}
\subsection{Database Interaction Summary}
	The choice of which lanugage with which we will use to interact with our database is an important decision.
	We want our projec to be very maintainable, both in the language\'s support and community, and the code we create with the language.
	For our project we want/don\'t... want lots of abstraction and are looking to use an old/modern... language. Because of these wants, we have chosen to use ... as our database interaction language.
	\subsection{Rails}
		Ruby on Rails is a flexible and expressive language that has been at the foreground of database interaction languages in recent years.
		There are a lot of protocols for how to implement web features in Rails. Rails is opinionated, meaning most features have a standard way of being created.
		This has positives and negatives; making implementing forms or buttons standardized makes Rails easily maintainable, but also doesn't allow the coder as much flexibility in designing their own functions.
		Problems arise when Rails has no opinion on how to implement a web feature, then the style of \"one size fits all\" in Rails doesn't work well. Rails is also very slow in comparison with Node or Php.
		For our purposes, Rails would be great for developing our project.
		We want our project to be as maintainable as possible, so it\'s a toss up on whether Rails or Node will be better.
		Both are anticipated to
		\\ \\
		\textbf{Pros: Maintainable, clean, protocols for everything}
		\\
		\textbf{Cons: A lot of abstraction (possibly too much depending on your purposes), not a defined way to do everything, way slow in comparison with other languages.}
		\\
		Sources: \\
		https://medium.com/@TechMagic/nodejs-vs-ruby-on-rails-comparison-2017-which-is-the-best-for-web-development-9aae7a3f08bf \\
		http://railsapps.github.io/what-is-ruby-rails.html \\ \\

	\subsection{Node.js}
			Node.js is a language you can use for full stack development.
			It allows Javascript to be run on the server side.
			Node is easy to learn, but unlike Rails, you have to build everything from the ground up.
			With little abstraction, you have total control of how features are implemented but development takes much longer.
			Node updates are often not backward compatible.
			Node can handle multiple requests at the same time and is way better at handling concurrent requests than Ruby.
			This means the app takes up less RAM and functions faster.
			Node has a high market demand, so for the intents of our project it would be very maintainable when we hand off our project to Ninkasi to expand.
			\\ \\
			\textbf{Pros: You can build a full stack with one language, easy to learn, strong community, actively developed language.}
			\\
			\textbf{Cons: You have to build every feature from the ground up, often backward incompatible updates.}
			\\
			Sources: \\
			https://medium.com/@TechMagic/nodejs-vs-ruby-on-rails-comparison-2017-which-is-the-best-for-web-development-9aae7a3f08bf \\
			https://www.netguru.co/blog/pros-cons-use-node.js-backend \\

	\subsection{Php}
		Php has the benefit of being around for a long time (infoworld).
		This language has a lot of longevity and is constantly updated. It has a large open source community and lots of support, so maintainability would be high(medium).
		The language is inconsistent and clunky, so this would diminish our development ease.
		There is a lack of abstraction, similar to Node.js, and often has repetitive code (railsapps), something Rails would reduce.
		For the ease of our coding, php would not work well in this project.
		\\ \\
		\textbf{Pros: Huge ecosystem and large open source community, always being updated, commonly used in industry for a long time.}
		\\
		\textbf{Cons: Very old, language is inconsistent, clunky, lack of abstraction.}
		\\
		Sources: \\
		http://railsapps.github.io/what-is-ruby-rails.html \\
		https://www.infoworld.com/article/2852329/php/reasons-for-developers-to-love-hate-php.html \\
		https://medium.com/@smartgamma/what-are-the-pros-and-cons-of-using-php-490553ed8ff2 \\

%\bibliography{references}{}
%\bibliographystyle{plain}

\end{document}
