% Midterm Progress Report
% CS 463 - CS Senior Capstone
% Spring 2018
% Authors: Connor Christensen, Lily Shellhammer, William Buffum
\documentclass[draftclsnofoot,onecolumn,letterpaper,10pt,compsoc]{IEEEtran}

% Packaging
\usepackage{geometry}
\usepackage{hyperref}
\usepackage{titling}
\usepackage{color}
\usepackage{listings}
\usepackage{cite}
\usepackage{pdfpages}
\usepackage{pdflscape}
\usepackage{url}
\usepackage{array}
\usepackage{graphicx}
\usepackage{subfig}

\definecolor{lightgray}{rgb}{0.95,0.95,0.95}
\definecolor{darkgray}{rgb}{0.5,0.5,0.5}
\definecolor{softred}{rgb}{0.7,0,0}

\lstdefinelanguage{JavaScript}{
  keywords={typeof, new, true, false, catch, function, return, null, catch, switch, var, if, in, while, do, else, case, break},
  keywordstyle=\color{blue}\bfseries,
  ndkeywords={class, export, boolean, throw, implements, import, this},
  ndkeywordstyle=\color{darkgray}\bfseries,
  identifierstyle=\color{black},
  sensitive=false,
  comment=[l]{//},
  morecomment=[s]{/*}{*/},
  commentstyle=\color{purple}\ttfamily,
  stringstyle=\color{red}\ttfamily,
  morestring=[b]',
  morestring=[b]"
}

\lstset{
   language=JavaScript,
   backgroundcolor=\color{lightgray},
   extendedchars=true,
   basicstyle=\footnotesize\ttfamily,
   showstringspaces=false,
   showspaces=false,
   numbers=left,
   numberstyle=\footnotesize,
   numbersep=9pt,
   tabsize=2,
   breaklines=true,
   showtabs=false,
   captionpos=b
}

\lstdefinelanguage{Stylus}{
  keywords={@media},
  keywordstyle=\color{blue}\bfseries,
  ndkeywords={px},
  ndkeywordstyle=\color{softred}\bfseries,
  identifierstyle=\color{black},
  sensitive=false,
  comment=[l]{//},
  morecomment=[s]{/*}{*/},
  commentstyle=\color{purple}\ttfamily,
  stringstyle=\color{red}\ttfamily,
  morestring=[b]',
  morestring=[b]"
}

\lstset{
   language=Stylus,
   backgroundcolor=\color{lightgray},
   extendedchars=true,
   basicstyle=\footnotesize\ttfamily,
   showstringspaces=false,
   showspaces=false,
   numbers=left,
   numberstyle=\footnotesize,
   numbersep=9pt,
   tabsize=2,
   breaklines=true,
   showtabs=false,
   captionpos=b
}

% Paper type
\geometry{letterpaper, margin=.75in}

% Title page
\title{CS 463 - CS Senior Capstone
	\\Spring 2017
	\\Midterm Progress Report
}

\author{
	Connor I. Christensen \\
	\texttt{chriconn@oregonstate.edu}
	\\
	Lily M. Shellhammer \\
	\texttt{shellhal@oregonstate.edu}
	\\
	William B. Buffum \\
	\texttt{buffumw@oregonstate.edu}
}

\begin{document}
\begin{titlingpage}
    \maketitle
    \begin{abstract}
			Ninkasi Brewing Company is based in Eugene, Oregon, producing and distributing nearly 100,000 barrels of beer each year across the United States and Canada.
			Ninkasi currently tracks brewery data using digital spreadsheets, a laborious, time consuming, and error prone process.
			Quality brewing requires the company to be detail-oriented, organize its data and provide timely actions in the brewing process.
			In order to maintain good quality control in their product and give the company room to scale in its production, our team has been tasked with creating software that will improve the process of entering, storing and accessing data related to the brewing process.
			This document examines work completed on the project to this date.
			At this point in the middle of the winter term, our product has reached an alpha state and has a working interface with no database connections.
			//TODO : ABOUT THIS REPORT
			\\
			\textbf{Keywords:} Brewing, Operations Management, Web App
    \end{abstract}
		\pagebreak
		\tableofcontents
\end{titlingpage}

% briefly recaps the project purposes and goals
% describes where you are currently on the project
% describes what you have left to do
% describes any problems that have impeded your progress, with any solutions you have
% includes particularly interesting pieces of code (if coding is involved)
    % Example set language to javascript
    % \lstset{language=JavaScript}
    % Example include a file with code in it
    % \lstinputlisting{mobile.js}
% includes images of your project -- screen shots, photos, whatever is appropriate
    % Example include an image
    % \centerline{\includegraphics[height=5cm]{screenshots/stylus.png}}

\section{Introduction}
Our team has created a data management web app for Ninkasi Brewing Company.
Ninkasi is a large craft brewery located in Eugene, Oregon that experienced exponential growth in recent years.
The brewing capacity has grown but the data managment technology remains antiquated.
Their outdated data entry system needed to be replaced, and our capstone team worked to create an alternative system.
In the last few weeks, we have finished a beta version of a web app that will serve as an efficient, modern data managment system.
\subsection{Purpose}
The purpose of this document is to outline what we have completed in the first half of our spring term.
We have finished the beta version of product and outline it's progress and challenges in this document.
\subsection{Scope}
This document outlines the role of each member of the team and their progress.
It also discusses future work and interesting code snippets.
\subsection{Overview}
We have reached beta version of our web app and sent a trial version to our client to test out in the brewery.
Our UI and database are nearly complete.
The functionality to check tank statuses, view batch history, enter cellaring data, and view graphs of data over time have all been implemented and have few bugs.
With the exception of a few bugs, we have met our project goals and some of our stretch goals.
\section{Connor Christensen}
\subsection{Current position in the project timeline}
\subsection{What is left to do}
\subsection{Problems encountered and solutions to those problems}
\subsection{Particularly interesting pieces of code}


\section{Lily Shellhammer}
\subsection{Describe where you are currently on the project}
I have reached beta version for the middle stack.
With the exception of a two bugs, our project is essentially finished.
At the end of last term, we had a few pillars of the site's javascript finished, but we were far from giving our client access to our site.
Since then, I (with some help from my teammates) have finished the following:
\begin{itemize}
\item Submit a new recipe on the admin page
\item Change tanks and actions input to drop down menus
\item Pull most recent data from that tank and auto-populate the form
\item Pull most recent data from tank for tank-info page
\item Edit tank status on the admin page
\item Show given names and airport codes instead of database ids
\item Add new form options to submissions (pressure, bright)
\item Have boxes display actions and colors correctly
\item Create tasks on admin page
\item Take out system of relying on version numbers and compare most recent values
\item Add labels to data entry form boxes
\end{itemize}
The majority of the work this term has been focused on the data entry, tank monitoring, and individual tank view pages.
These are important pages not only because they are the most used, but because they have the most complicated code.
In all of these pages we are pulling the most recent information for a batch given a tank number.
We have to pull the tank information, pull the batch infomration, then find the batch associated with the tank.
Then we pull the batch contents version information, find the batch associated with the tank, and pull the reading with the most recent datestamp.
Then we iterate through the task and actions pages to find if there are any actions needed on specific batches.
\\
Firstly, on the data entry page we utilize this process to pull most recent data so that the brewers don't have to retype measurements that don't often change.
On our tank monitoring page, this process pulls the most recent measurements and actions so that we can display them on small boxes, giving the brewer and overview of measurements and batch statuses.
And finally on our individual tank view page, the most recent readings for every data point are displayed using this process.
\\

\subsection{Describe what you have left to do}
We have two main issues to fix. One is updating the actions of a tank.
We currently can only set the task to a certain action and cannot change that entry.
By adding another action we are able to change the color of the tank as it is associated with a new action, but we cannot edit an old action.
\\
The other issue is submitting a recipe.
We currently submit only a JSON string with the ingredient names specified, not the rates.
This is a difficult fix for me as I don't fully understand how to compound objects or append to them and am struggling the right information online.
\subsection{Describe any problems that have impeded your progress, with any solutions}
Naming was the largest bug creator for me.
There are names in the database that were specifically chosen so as to streamline the whole system's naming, yet conflict with terminology we were originally using.
Because the code has been evolving and changing names with the updating of our database, we often got in trouble where we hadn't updated all the names, yet the old names corresponded to new values so no errors were thrown.
\subsection{Include particularly interesting pieces of code}
One cool feature we implemented is on our data submission page, where we have a drop down menu for the tanks.
When a tank is chosen, we implemented a feature that auto-populates the form with the most recent data taken on those tanks.
This is because there are measurements that don't change much, so we let the brewer use old data and not have to retype everything.
\begin{figure}
  \centering
  \includegraphics[height=10cm]{screenshots/lily/lilysCode.png}
  \caption{The code used to auto-populate the form with the most recent data}
\end{figure}


\section{William Buffum}
\subsection{Current position in the project timeline}
\subsection{What is left to do}
\subsection{Problems encountered and solutions to those problems}
\subsection{Particularly interesting pieces of code}

\section{Conclusion}

\subsection{Images of the product}

\begin{figure}
  \centering
  \includegraphics[height=10cm]{screenshots/desktop/login.png}
  \caption{Login page on desktop}
\end{figure}
\begin{figure}
  \centering
  \centerline{\includegraphics[height=10cm]{screenshots/desktop/admin.png}}
  \caption{Admin page on desktop}
\end{figure}
\begin{figure}
  \centering
  \centerline{\includegraphics[height=10cm]{screenshots/desktop/home.png}}
  \caption{Home page on desktop}
\end{figure}
\begin{figure}
  \centering
  \centerline{\includegraphics[height=10cm]{screenshots/desktop/home_filled.png}}
  \caption{Home page on desktop with filled content in the data entry component}
\end{figure}
\begin{figure}
  \centering
  \centerline{\includegraphics[height=10cm]{screenshots/desktop/tank_info.png}}
  \caption{Tank info page on desktop}
\end{figure}
\begin{figure}
  \centering
  \centerline{\includegraphics[height=10cm]{screenshots/desktop/data_submission.png}}
  \caption{Data submission page on desktop}
\end{figure}

\begin{figure}
  \centering
  \centerline{\includegraphics[height=10cm]{screenshots/desktop/batch_history.png}}
  \caption{Batch history page on desktop}
\end{figure}
\begin{figure}
  \centering
  \centerline{\includegraphics[height=10cm]{screenshots/desktop/csv.png}}
  \caption{The downloaded information from the batch history page in a csv}
\end{figure}


\begin{figure}%
    \centering
    \subfloat[Login]{{\includegraphics[height=13cm]{screenshots/mobile/login.png}}}%
    \qquad
    \subfloat[Home]{{\includegraphics[height=13cm]{screenshots/mobile/home.png}}}%
    \caption{Mobile - Login and Home page}%
\end{figure}

\begin{figure}%
    \centering
    \subfloat[Tank Monitoring]{{\includegraphics[height=13cm]{screenshots/mobile/tank_monitoring.png}}}%
    \qquad
    \subfloat[Tank Info]{{\includegraphics[height=13cm]{screenshots/mobile/tank_info.png}}}%
    \caption{Mobile - Tank monitoring and more info page}%
\end{figure}

\begin{figure}
	\centering
	\includegraphics[height=13cm]{screenshots/mobile/admin.png}
  \caption{Admin page on mobile}
\end{figure}

\begin{figure}%
    \centering
    \subfloat[Data Entry]{{\includegraphics[height=13cm]{screenshots/mobile/data_entry.png}}}%
    \qquad
    \subfloat[Submission]{{\includegraphics[height=13cm]{screenshots/mobile/submission.png}}}
    \caption{Mobile - Data entry and submission pages}%
\end{figure}

\end{document}
